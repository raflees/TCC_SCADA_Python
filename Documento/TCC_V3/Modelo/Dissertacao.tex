% !TeX spellcheck = pt_BR
% Observação lista de abreviaturas e símbolos:
% faz-se necessário executar os seguintes comandos no terminal (na pasta
% ue estão os arquivos) 
% makeindex -s pei.ist -o Dissertacao.lab Dissertacao.abx
% makeindex -s pei.ist -o Dissertacao.los Dissertacao.syx

\documentclass[msc]{pei}
\usepackage{cmap}
\usepackage[utf8]{inputenc}
\usepackage[T1]{fontenc}
\usepackage{amsmath, amsfonts, amssymb, amstext, amsthm, mathtools, xfrac}
\usepackage[alf,abnt-etal-list=0]{abntex2cite}
\usepackage[brazil,nohints]{minitoc}
\usepackage{listings}
\usepackage{capitulos}
\usepackage{color}
\usepackage[font=footnotesize]{caption}
\usepackage{subcaption}
\usepackage{tabulary, array}
\usepackage{rotating}
\usepackage{graphicx}
\usepackage{icomma}
\usepackage{epigraph}
\usepackage{pdfpages}
\usepackage{hyperref}
\usepackage{trivfloat}
\usepackage{tikz}
\usepackage{multirow}
\usepackage{placeins}
\usepackage{placeins}
\usepackage{makecell}
\usepackage{threeparttable}


\newcommand{\figref}[1]{Figura~\ref{#1}}
\newcommand{\eqnref}[1]{Equa\c{c}\~{a}o~\eqref{#1}}
\newcommand{\tabref}[1]{Tabela~\ref{#1}}
\newcommand{\quadref}[1]{Quadro~\ref{#1}}
\newcommand{\exref}[1]{Ap\^{e}ndice~\ref{#1}}

%\newcommand\addtotoc[1]{\refstepcounter{dummy}\addcontentsline{toc}{chapter}{#1}}

\renewcommand{\lstlistingname}{Código}
\renewcommand{\lstlistlistingname}{Lista de Códigos}
\renewcommand\cellalign{cc}

\setcounter{MaxMatrixCols}{25}
\captionsetup{belowskip=6pt,aboveskip=4pt}
\setlength{\extrarowheight}{1.05pt}

%syntax highlighthing for EMSO codes
\lstdefinelanguage{EMSO}{
  morekeywords={
      Real,Boolean,Text,Integer,CalcObject,Model,FlowSheet,using, in, ext, out, as, to, end, switch, case, Switcher, time, diff, sin, cos, tan, exp, ln, log, if, else, asin, sqrt, outer, Plugin, Brief, Default, switchto, when, Optimization, Estimation, Reconciliation, CaseStudy, Sensitivity, Filter, GrossErrorTests, Significance, ObjectiveFunction, Global, Nodal, Measurements, RunTests, Statistics, BiLateral, Lower, Upper, Dynamic, false, true, TimeStart, TimeStep, TimeEnd, TimeUnit, final, Unit, DisplayUnit, sum, for, SET, DEVICES, PARAMETERS, VARIABLES, CONNECTIONS, GUESS, MINIMIZE, FREE, MAXIMIZE, ESTIMATE, EXPERIMENTS, RECONCILE, VARY, RESPONSE, DAESolver, INITIAL, EQUATIONS, SPECIFY, OPTIONS, ATTRIBUTES, Pallete, Info, MaxIterations, NLPSolver, RelativeAccuracy, File, Hessian_approximation, },
  sensitive = true,
  morecomment=[l]{\#},
  morecomment=[s]{\#*}{*\#},
  morestring=[b]",
  morestring=[b]'
}
\lstset{
  basicstyle=\fontfamily{pcr}\fontseries{m}\selectfont\footnotesize,
  commentstyle=\color[rgb]{0.3,0.6,0}\itshape,
  keywordstyle=\color{blue}\bfseries,
  stringstyle=\color[rgb]{0.5,0,0.5}\itshape,
  showstringspaces=false,
  numbers=left,
  numberstyle=\color[rgb]{0,0.5,0.5}\fontfamily{pcr}\fontseries{m}\selectfont\tiny,
  numberblanklines=true,
  showlines=false,
  belowskip=\bigskipamount{},
  breaklines=true,
  %stepnumber=2,
  tabsize=6,
  %extendedchars=true,
  %float=h,
  frame=tb
}

\everymath{\displaystyle}
\allowdisplaybreaks

\theoremstyle{definition}
\newtheorem{definition}{Defini\c{c}\~{a}o}[chapter]

\makelosymbols
\makeloabbreviations

% Quadro
\trivfloat{quadro}
\floatstyle{plaintop} % Forçar posição da legenda
\restylefloat{quadro} % Forçar posição da legenda
\renewcommand{\listquadroname}{Lista de Quadros} % Forçar texto na Lista de Quadros


\addto\captionsbrazil{
    %% ajusta nomes padroes do babel
    \renewcommand{\bibname}{Refer\^encias}
    %\renewcommand{\indexname}{\’Indice}
    %\renewcommand{\listfigurename}{Lista de ilustra\c{c}\~{o}es}
    %\renewcommand{\listtablename}{Lista de tabelas}
    %% ajusta nomes usados com a macro \autoref
    %\renewcommand{\pageautorefname}{p\’agina}
    %\renewcommand{\sectionautorefname}{se{\c c}\~ao}
    %\renewcommand{\subsectionautorefname}{subse{\c c}\~ao}
    %\renewcommand{\paragraphautorefname}{par\’agrafo}
    %\renewcommand{\subsubsectionautorefname}{subse{\c c}\~ao}
}


\begin{document}
	
\title{TÍTULO}
\foreigntitle{TITLE}
\author{Rafael Guimarães de Araújo}{Lucena}
\advisor{Prof. Dr.}{Daniel Diniz}{Santana}{D.Sc.}
%\coadvisor{Prof. Dr.}{NOME SOBRENOME}{ÚLTIMO SOBRENOME}{D.Sc.}
%\examiner{Prof.}{Argimiro Resende Secchi}{D.Sc.}
%\examiner{Prof.}{Leizer Schnitman}{D.Sc.}
%\examiner{}{Pleycienne Trajano Ribeiro}{D.Sc.}
\department{PEI}
\date{12}{2020}
\keyword{Sistema Supervisório}
\keyword{open-source}
\keyword{SCADA}
\keyword{Python}
\cdd{S2320}{511}

%\includepdf{CAPA_DISSERTACAO_PEI_DANIEL_DINIZ_SANTANA_FRENTE.pdf}

\maketitle

%\frontmatter

%\includepdf{FOLHA_APROVACAO.pdf}
%\dedication{Dedicatória}

%\chapter*{Agradecimentos}

%Agradecimentos


%\epigraph{\textit{``Nunca tenha certeza de nada, porque a sabedoria começa com a dúvida.''}}{Freud}

%\epigraph{\textit{``A única coisa que me espera é exatamente o inesperado.''}}{Clarice Lispector}

\begin{abstract}

\noindent Resumo.


\noindent \textbf{Palavras-chave.} palavra-chave 1, palavra-chave 2, palavra-chave 3

\end{abstract}

\begin{foreignabstract}

\noindent abstract

\noindent \textbf{Keywords.} keyword 1, keyword 2, keyword 3

\end{foreignabstract}

\dominitoc
\tableofcontents
\listoffigures
\listoftables

%inclusão da lista de quadros no sumário
\newpage
\phantomsection
\addcontentsline{toc}{chapter}{\listquadroname} \mtcaddchapter
\listofquadros
%fim da inclusão da lista de quadros

%\lstlistoflistings
\printlosymbols
\printloabbreviations

\setcounter{mtc}{5}
\mainmatter


\chapter{Exemplo} \label{Chap:Exemplo}

\section{Seção de exemplo}

Aqui você escreve!

Para incluir uma abreviatura use: \abbrev{Abreviatura}{Descrição}, assim que for utilizada pela primeira vez

Para incluir um símbolo use: \symbl{$simbolo$}{descrição}, assim que ele for utilizado.












\backmatter

% \bibliographystyle{abnt-alf}
% INCLUA AQUI O SEU ARQUIVO DE REFERÊNCIA
%\bibliography{References_rev02.bib}

\appendix
\addcontentsline{toc}{chapter}{Anexos e Apêndices}

\chapter{Exemplo} \label{Chap:Exemplo}

\section{Código do arduino para Controle de Tanque com Área Variável}

\begin{lstlisting}

#include <stdlib.h>

//Variables
float dh1dt;
float dh2dt;
float h1;
float h2;
float u1;
float u2;
float t0;
float t;

//Parameters
float pi = 3.1415926535897932384626433832795;
float B = 1.;
float A = 4.;
float hM = 4.;
float gamma = ((A/2.)-(B/2.))/2.;
float k = 0.001;
float g = 9.8;
float rho = 1000;
float c = k*pow(rho*g,0.5);
//sample time in miliseconds
float tstep = 100;

//Espera por um sinal vindo do computador
void wait_for_comm() {
	while (true) {
		if (Serial.available() > 0) {
			break;
		}
	}
	clearSerial();
	return;
}

//Limpa a porta seria para nao atrapalhar futuras leituras
void clearSerial() {
	char c;
	while(Serial.available() > 0)
		c = Serial.read();
	return;
}

void setup() {
	// put your setup code here, to run once:
	Serial.begin(9600);
	
	h1 = 1.0;
	h2 = 2.0;
	u1 = 0;
	u2 = 0;
	
	wait_for_comm();
	t0 = millis();
}

void loop() {
// put your main code here, to run repeatedly:
	if (Serial.available()) {
		Serial.flush();
		u1 = Serial.parseFloat();
		u2 = Serial.parseFloat();
	}

	//Limitacao da entrada
	if (u1 > 1) u1 = 1;
	if (u1 < 0) u1 = 0;
	
	if (u2 > 1) u2 = 1;
	if (u2 < 0) u2 = 0;
	
	t = millis() - t0;
	t0 = millis();
	for (int i = 0; i < ceil(t/tstep); i++) {
		dh1dt = (1./(pi*pow(gamma,2)*pow(h1+(B/2)/gamma,2))*(u1+c*pow(h2,0.5)-c*pow(h1,0.5)));
		dh2dt = (1./(pi*pow(gamma,2)*pow(h2+(B/2)/gamma,2))*(u2-c*pow(h2,0.5)));
		
		h1 += dh1dt*tstep/1000;
		h2 += dh2dt*tstep/1000;
	
		//Limitacao dos estados
		if (h2<0){
			h2 = 0;
		} else if(h2>hM){
			h2 = hM;
		}
	
		if (h1<0.){
			h1 = 0;
		} else if(h1>hM){
			h1 = hM;
		}
	}
	
	Serial.print(t0/1000);
	Serial.print('\t');
	Serial.print(h1, 2);//+(float) random(-1,1)/80.,2);
	Serial.print('\t');
	Serial.print(h2, 2);//+(float) random(-1,1)/80.,2);
	Serial.print('\t');
	Serial.print(u1,2);
	Serial.print('\t');
	Serial.println(u2,2);
}
\end{lstlisting}

%\includepdf{CAPA_DISSERTACAO_PEI_DANIEL_DINIZ_SANTANA_FUNDO.pdf}

\end{document}
