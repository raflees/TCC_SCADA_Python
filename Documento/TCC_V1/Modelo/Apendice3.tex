\chapter{Códigos} \label{Chap:Apendice3}

\section{Código do arduino para Controle de Tanque com Área Variável}

\begin{lstlisting}

#include <stdlib.h>

//Variables
float dh1dt;
float dh2dt;
float h1;
float h2;
float u1;
float u2;
float t0;
float t;

//Parameters
float pi = 3.1415926535897932384626433832795;
float B = 1.;
float A = 4.;
float hM = 4.;
float gamma = ((A/2.)-(B/2.))/2.;
float k = 0.001;
float g = 9.8;
float rho = 1000;
float c = k*pow(rho*g,0.5);
//sample time in miliseconds
float tstep = 100;

//Espera por um sinal vindo do computador
void wait_for_comm() {
	while (true) {
		if (Serial.available() > 0) {
			break;
		}
	}
	clearSerial();
	return;
}

//Limpa a porta seria para nao atrapalhar futuras leituras
void clearSerial() {
	char c;
	while(Serial.available() > 0)
		c = Serial.read();
	return;
}

void setup() {
	// put your setup code here, to run once:
	Serial.begin(9600);
	
	h1 = 1.0;
	h2 = 2.0;
	u1 = 0;
	u2 = 0;
	
	wait_for_comm();
	t0 = millis();
}

void loop() {
// put your main code here, to run repeatedly:
	if (Serial.available()) {
		Serial.flush();
		u1 = Serial.parseFloat();
		u2 = Serial.parseFloat();
	}

	//Limitacao da entrada
	if (u1 > 1) u1 = 1;
	if (u1 < 0) u1 = 0;
	
	if (u2 > 1) u2 = 1;
	if (u2 < 0) u2 = 0;
	
	t = millis() - t0;
	t0 = millis();
	for (int i = 0; i < ceil(t/tstep); i++) {
		dh1dt = (1./(pi*pow(gamma,2)*pow(h1+(B/2)/gamma,2))*(u1+c*pow(h2,0.5)-c*pow(h1,0.5)));
		dh2dt = (1./(pi*pow(gamma,2)*pow(h2+(B/2)/gamma,2))*(u2-c*pow(h2,0.5)));
		
		h1 += dh1dt*tstep/1000;
		h2 += dh2dt*tstep/1000;
	
		//Limitacao dos estados
		if (h2<0){
			h2 = 0;
		} else if(h2>hM){
			h2 = hM;
		}
	
		if (h1<0.){
			h1 = 0;
		} else if(h1>hM){
			h1 = hM;
		}
	}
	
	Serial.print(t0/1000);
	Serial.print('\t');
	Serial.print(h1, 2);//+(float) random(-1,1)/80.,2);
	Serial.print('\t');
	Serial.print(h2, 2);//+(float) random(-1,1)/80.,2);
	Serial.print('\t');
	Serial.print(u1,2);
	Serial.print('\t');
	Serial.println(u2,2);
}
\end{lstlisting}