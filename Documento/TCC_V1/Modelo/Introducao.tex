\chapter{Introdução} \label{Chap:Introdução}

\section{Contextualização e Justificativa}

O monitoramento remoto de processos é uma área da tecnologia surgida no século XX, quando os sistemas de controle passaram a ser aplicados na indústria. Segundo \citeonline{junior2019}, as interfaces homem-máquina passadas se baseavam em relés que acendiam lâmpadas e quadros sinóticos, representando alertas e mostradores. Como avanço dos meios de transmissão de dados, displays digitais e comunicação entre diferentes sistemas, os computadores se modernizaram o suficiente para tomar o espaço industrial, no que diz respeito ao controle e monitoramento de processos.

Neste âmbito da modernização industrial, surgiram os sistemas SCADA
\textit{\textup{(}Supervisory Control And Data Acquisition\textup{)}}, também chamados de sistemas supervisórios. Dentre alguns exemplos de \textit{software} da área, se encontram o Elipse E3 e o SIMATIC WinCC, que podem ser executados em computadores comuns e se comunicam com outros dispositivos eletrônicos via protocolos de comunicação como Modbus e Profibus (\cite{Elipse2012}, \cite{Siemens1999}). Com isso, podem enviar e receber dados de controladores de uma área industrial, e mostrar variáveis de processo em um monitor através de diversos recursos visuais, como gráficos, símbolos e tabelas. Esta dinâmica facilitam a um operador entender rapidamente o estado de suas máquinas e identificar tendências de falhas.

\citeonline{Martins2007} afirma que os dados adiquiridos de controladores espalhados numa área industrial podem ser manipulados, de forma a servir de insumos para definição de \emph{setpoints}. Num processo de aquecimento, por exemplo, a medição de temperaturas baixas pode enviar ao sistema de controle um setpoint maior que o desejado, com o objetivo forçar um pouco mais o sistema de controle, ao menos até atingir valores mais próximos do desejado.

Segundo \citeonline{junior2019}, o valor de um sistema SCADA está relacionado com o conceito de \textit{software} aberto. O autor descreve este conceito, como um \textit{software} o mais compatível possível com os hardwares com os quais se comunicaria e com os diferentes sistemas operacionais que podem executá-lo. Além disto, o \emph{software} deve ser modular, executado em módulos, de forma que o mal-funcionamento de uma de suas partes não impacte negativamente a execução das outras. Por fim, deve também ser escalável, e permitir a agregação de novos equipamentos e novas funcionalidades.

Frente à sua importância na área da automação, aspirantes a engenheiros devem se familiarizar com esta tecnologia. Atualmente existem ferramentas gratuitas focadas neste propósito, geralmente aplicadas no meio acadêmico, como descrita por \citeonline{Moraes2016}. Os chamados \emph{software} \emph{open source} são gratuitos, e seus códigos-fontes podem ser compartilhados, modificados e aplicados à vontade.

O programa ScadaBR, por exemplo, trata-se justamente de um sistema SCADA gratuito, construído em um servidor web \cite{silva2013}. O supervisório Elipse E3 também fornece uma versão de demonstração, com algumas limitações em relação à versão paga. Por fim, existem linguagens de programação \emph{open source}, como Python e C++, que possuem diversos materiais didáticos gratuitos disponíveis \emph{online}.

Tratando-se de plataformas pagas, exemplifica-se o \textit{software} de modelagens computacionais, utilizado na engenharia, chamado MATLab\textsuperscript{\tiny \textregistered}. Ele oferece um pacote voltado para estudantes que inclui o \emph{Simulink} para modelagem, simulação de processos, entre outras funcionalidades, e ferramentas para \emph{machine learning}. O valor deste conjunto em 03/12/20 era de USD 55,00 \cite{MathWorks}. No que concerne a supervisórios, o WinCC, fabricado pela Siemens é vendido por um de seus distribuidores, na versão Flexible por mais que R\$ 14.000,00 \cite{SHMR}.
	
Dados os valores de \textit{software} de engenharia, e a difusão de tecnologias \emph{open source}, existe uma necessidade e oportunidade de instituições de ensino aderirem à utilização de ferramentas gratuitas, desenvovidas pelos próprios estudantes. As vantagens não são somente financeiras, pois a construção destas ferramentas acrescenta profissionalmente a todos os envolvidos.

\section{Objetivos}

\subsection{Objetivos Gerais}

Desenvolver um sistema supervisório didático e gratuito capaz de se comunicar com dispositivos externos para construção de gráficos em sua interface e simulações em tempo real.

\subsection{Objetivos Específicos}

Com o foco no objetivo supracitado, foram levantados os seguintes objetivos específico direcionados desenvolvimento do supervisório didático:

\begin{itemize}
	\item Definir os requisitos de comunicação;
	\item Projetar e desenvolver a interface gráfica e os eventos que coordenam o funcionamento do sistema;
	\item Mapear e documentar o \textit{software} para facilitar futuras melhorias;
	\item Validar o funcionamento do supervisório com uma aplicação prática.
\end{itemize}

\section{Estrutura do Trabalho}

Este trabalho foi dividido em:

\begin{enumerate}
	\item No capítulo 1 o tema principal foi contextualizado de acordo com a tecnologia atual, o problema a ser solucionado foi apresentado e os objetivos foram esclarecidos.
	\item O capítulo 2 apresenta os conceitos básicos relativos aos assuntos que cercam o tema principal deste trabalho, contendo uma descrição curta dos sistemas SCADA existentes, das tecnologias que podem ser utilizadas para a criação de um supervisório e finalmente um resumo do método selecionado para a construção da ferramenta.
	\item O capítulo 3 trata da programação do supervisório em si, mostrando sua arquitetura e convenções empregadas com figuras e esquemas.
	\item O capítulo 4 toma o sistema pronto e o emprega em uma aplicação prática, com o intuito de validar seu funcionamento e exemplificar como o mesmo pode contribuir no ambiente acadêmico da universidade.
	\item No capítulo 5 este trabalho se encerra em forma de um texto conclusivo, onde, entre outros assuntos, são abordadas possíveis futuras melhorias ao sistema criado.
\end{enumerate}