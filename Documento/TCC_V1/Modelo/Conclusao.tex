\chapter{Conclusão}

Este trabalho teve como objetivo principal apresentar e testar um sistema supervisório didático construído em Python, que permitisse aos usuários monitorar valores recebidos pela porta serial da máquina na qual fosse executado, e ainda a criarem suas próprias séries de dados. De outra perspectiva, este trabalho almejou fomentar o emprego de tecnologias \textit{open source} no ambiente acadêmico, e encorajar os estudantes do ensino superior a se familiarizarem com linguagens de programação, sobretudo Pyhton, que vem ganhando adeptos no cenário atual.

Frente aos objetivos levantados, pode-se afirmar que o presente trabalho alcançou seu propósito, como exemplificado nos casos de teste. Utilizando a ferramenta desenvolvida, o estudante pode criar um sistema físico controlado por microcontrolador, e monitorar em tempo real o comportamento deste sistema. Ainda, pode salvar o gráfico da resposta no programa e compará-lo com outros comportamentos esperados, modelados também dentro do programa.

A seção de apresentação do sistema abordou algumas técnicas adotadas na programação do \textit{software}, e representa um guia para construções de GUIs utilizando a bilbioteca PyQt. Apesar de haver diversos vídeos, exemplos e tutoriais na internet tratando desta biblioteca, este trabalho se destaca não só a aplicar Python no âmbito da automação, como a enriquecer a interface com outros recursos da linguagem, como a serialização de objetos e a paralelização do código em threads.

Espera-se que o produto entregue neste trabalho de conclusão de curso seja de grande valia para os estudantes, bem como para a universidade.

\section{Trabalhos futuros}

Por se tratar de um código aberto, qualquer interessado tem a possibilidade de alterar os scripts como desejar, incluindo novas funcionalidades ou modificando as existentes. Acredita-se que, por ter sido construído em uma linguagem \textit{open source}, o supervisório didático deixa muitas possibilidades de melhoria, como as citadas a seguir, ordenadas por complexidade:

\begin{enumerate}
	\item Permitir que o usuário exporte os resultados em diferentes formatos de arquivos;
	\item Adicionar um módulo no programa responsável pela escrita em um banco de dados das séries registradas, assim como feito em sistemas SCADA industriais;
	\item Criar, separadamente do supervisório, um objeto que emule um CLP;
	\item Incluir novos meios de comunicação com periféricos, como por bluetooth ou rede Wi-Fi;
	\item Acrescentar novas formas de simulação de sistemas, não somente por funções de transferência simples, mas por equações descritivas, ou até mesmo um ambiente que monte um diagrama de blocos, como implementado por ferramentas como o MATLab\textsuperscript{\tiny \textregistered} e o SciLab;
	\item Incluir ferramentas de análise de processos como perditor de Smith e estimadores e avaliadores de estado.
\end{enumerate}