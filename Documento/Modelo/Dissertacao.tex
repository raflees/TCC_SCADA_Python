% !TeX spellcheck = pt_BR
% Observação lista de abreviaturas e símbolos:
% faz-se necessário executar os seguintes comandos no terminal (na pasta
% ue estão os arquivos) 
% makeindex -s pei.ist -o Dissertacao.lab Dissertacao.abx
% makeindex -s pei.ist -o Dissertacao.los Dissertacao.syx

\documentclass[msc]{pei}
\usepackage{array}
\usepackage{cmap}
\usepackage[utf8]{inputenc}
\usepackage[T1]{fontenc}
\usepackage{amsmath, amsfonts, amssymb, amstext, amsthm, mathtools, xfrac}
\usepackage[alf,abnt-etal-list=0]{abntex2cite}
\usepackage[brazil,nohints]{minitoc}
\usepackage{listings}
\usepackage{capitulos}
\usepackage{color}
\usepackage[font=footnotesize]{caption}
\usepackage{subcaption}
\usepackage{tabulary, array}
\usepackage{rotating}
\usepackage{graphicx}
\usepackage{icomma}
\usepackage{epigraph}
\usepackage{pdfpages}
\usepackage{hyperref}
\usepackage{trivfloat}
\usepackage{tikz}
\usepackage{multirow}
\usepackage{placeins}
\usepackage{placeins}
\usepackage{makecell}
\usepackage{threeparttable}

\graphicspath{{imagens/}}
\newcommand{\figref}[1]{Figura~\ref{#1}}
\newcommand{\eqnref}[1]{Equa\c{c}\~{a}o~\eqref{#1}}
\newcommand{\tabref}[1]{Tabela~\ref{#1}}
\newcommand{\quadref}[1]{Quadro~\ref{#1}}
\newcommand{\exref}[1]{Ap\^{e}ndice~\ref{#1}}

%\newcommand\addtotoc[1]{\refstepcounter{dummy}\addcontentsline{toc}{chapter}{#1}}

\renewcommand{\lstlistingname}{Código}
\renewcommand{\lstlistlistingname}{Lista de Códigos}
\renewcommand\cellalign{cc}

\setcounter{MaxMatrixCols}{25}
\captionsetup{belowskip=6pt,aboveskip=4pt}
\setlength{\extrarowheight}{1.05pt}

%syntax highlighthing for EMSO codes
\lstdefinelanguage{EMSO}{
  morekeywords={
      Real,Boolean,Text,Integer,CalcObject,Model,FlowSheet,using, in, ext, out, as, to, end, switch, case, Switcher, time, diff, sin, cos, tan, exp, ln, log, if, else, asin, sqrt, outer, Plugin, Brief, Default, switchto, when, Optimization, Estimation, Reconciliation, CaseStudy, Sensitivity, Filter, GrossErrorTests, Significance, ObjectiveFunction, Global, Nodal, Measurements, RunTests, Statistics, BiLateral, Lower, Upper, Dynamic, false, true, TimeStart, TimeStep, TimeEnd, TimeUnit, final, Unit, DisplayUnit, sum, for, SET, DEVICES, PARAMETERS, VARIABLES, CONNECTIONS, GUESS, MINIMIZE, FREE, MAXIMIZE, ESTIMATE, EXPERIMENTS, RECONCILE, VARY, RESPONSE, DAESolver, INITIAL, EQUATIONS, SPECIFY, OPTIONS, ATTRIBUTES, Pallete, Info, MaxIterations, NLPSolver, RelativeAccuracy, File, Hessian_approximation, },
  sensitive = true,
  morecomment=[l]{\#},
  morecomment=[s]{\#*}{*\#},
  morestring=[b]",
  morestring=[b]'
}
\lstset{
  basicstyle=\fontfamily{pcr}\fontseries{m}\selectfont\footnotesize,
  commentstyle=\color[rgb]{0.3,0.6,0}\itshape,
  keywordstyle=\color{blue}\bfseries,
  stringstyle=\color[rgb]{0.5,0,0.5}\itshape,
  showstringspaces=false,
  numbers=left,
  numberstyle=\color[rgb]{0,0.5,0.5}\fontfamily{pcr}\fontseries{m}\selectfont\tiny,
  numberblanklines=true,
  showlines=false,
  belowskip=\bigskipamount{},
  breaklines=true,
  %stepnumber=2,
  tabsize=6,
  %extendedchars=true,
  %float=h,
  frame=tb
}

\everymath{\displaystyle}
\allowdisplaybreaks

\theoremstyle{definition}
\newtheorem{definition}{Defini\c{c}\~{a}o}[chapter]

\makelosymbols
\makeloabbreviations

% Quadro
\trivfloat{quadro}
\floatstyle{plaintop} % Forçar posição da legenda
\restylefloat{quadro} % Forçar posição da legenda
\renewcommand{\listquadroname}{Lista de Quadros} % Forçar texto na Lista de Quadros


\addto\captionsbrazil{
    %% ajusta nomes padroes do babel
    \renewcommand{\bibname}{Refer\^encias}
    %\renewcommand{\indexname}{\’Indice}
    %\renewcommand{\listfigurename}{Lista de ilustra\c{c}\~{o}es}
    %\renewcommand{\listtablename}{Lista de tabelas}
    %% ajusta nomes usados com a macro \autoref
    %\renewcommand{\pageautorefname}{p\’agina}
    %\renewcommand{\sectionautorefname}{se{\c c}\~ao}
    %\renewcommand{\subsectionautorefname}{subse{\c c}\~ao}
    %\renewcommand{\paragraphautorefname}{par\’agrafo}
    %\renewcommand{\subsubsectionautorefname}{subse{\c c}\~ao}
}


\begin{document}
	
\title{TÍTULO}
\foreigntitle{TITLE}
\author{RAFAEL GUIMARÃES}{LUCENA}
\advisor{Prof. Dr.}{NOME SOBRENOME}{ÚLTIMO SOBRENOME}{D.Sc.}
\coadvisor{Prof. Dr.}{NOME SOBRENOME}{ÚLTIMO SOBRENOME}{D.Sc.}
%\examiner{Prof.}{Argimiro Resende Secchi}{D.Sc.}
%\examiner{Prof.}{Leizer Schnitman}{D.Sc.}
%\examiner{}{Pleycienne Trajano Ribeiro}{D.Sc.}
\department{PEI}
\date{12}{2020}
\keyword{Sistema Supervisório}
\keyword{Python}
\keyword{Sistema Open Source}
\keyword{PyQt}
\cdd{S2320}{511}

%\includepdf{CAPA_DISSERTACAO_PEI_DANIEL_DINIZ_SANTANA_FRENTE.pdf}

\maketitle

\frontmatter

%\includepdf{FOLHA_APROVACAO.pdf}
\dedication{Dedicatória}

\chapter*{Agradecimentos}

Agradecimentos


\epigraph{\textit{``Nunca tenha certeza de nada, porque a sabedoria começa com a dúvida.''}}{Freud}

\epigraph{\textit{``A única coisa que me espera é exatamente o inesperado.''}}{Clarice Lispector}

\begin{abstract}

\noindent Resumo.


\noindent \textbf{Palavras-chave.} palavra-chave 1, palavra-chave 2, palavra-chave 3

\end{abstract}

\begin{foreignabstract}

\noindent abstract

\noindent \textbf{Keywords.} keyword 1, keyword 2, keyword 3

\end{foreignabstract}

\dominitoc
\tableofcontents
\listoffigures
\listoftables

%inclusão da lista de quadros no sumário
\newpage
\phantomsection
\addcontentsline{toc}{chapter}{\listquadroname} \mtcaddchapter
\listofquadros
%fim da inclusão da lista de quadros

%\lstlistoflistings
\printlosymbols
\printloabbreviations

\setcounter{mtc}{5}
\mainmatter


\chapter{Introdução} \label{Chap:Introdução}

\section{Contextualização e Justificativa}

O monitoramento remoto de processos é uma área da tecnologia surgida no século XX, quando os sistemas de controle passaram a ser aplicados na indústria. Segundo \citeonline{junior2019}, as interfaces homem-máquina passadas se baseavam em relés que acendiam lâmpadas e quadros sinóticos, representando alertas e mostradores. Como avanço dos meios de transmissão de dados, displays digitais e comunicação entre diferentes sistemas, os computadores se modernizaram o suficiente para tomar o espaço industrial, no que diz respeito ao controle e monitoramento de processos.

Neste âmbito da modernização industrial, surgiram os sistemas SCADA
\textit{\textup{(}Supervisory Control And Data Acquisition\textup{)}}, também chamados de sistemas supervisórios. Dentre alguns exemplos de \textit{software} da área, se encontram o Elipse E3 e o SIMATIC WinCC, que podem ser executados em computadores comuns e se comunicam com outros dispositivos eletrônicos via protocolos de comunicação como Modbus e Profibus (\cite{Elipse2012}, \cite{Siemens1999}). Com isso, podem enviar e receber dados de controladores de uma área industrial, e mostrar variáveis de processo em um monitor através de diversos recursos visuais, como gráficos, símbolos e tabelas. Esta dinâmica facilitam a um operador entender rapidamente o estado de suas máquinas e identificar tendências de falhas.

\citeonline{Martins2007} afirma que os dados adiquiridos de controladores espalhados numa área industrial podem ser manipulados, de forma a servir de insumos para definição de \emph{setpoints}. Num processo de aquecimento, por exemplo, a medição de temperaturas baixas pode enviar ao sistema de controle um setpoint maior que o desejado, com o objetivo forçar um pouco mais o sistema de controle, ao menos até atingir valores mais próximos do desejado.

Segundo \citeonline{junior2019}, o valor de um sistema SCADA está relacionado com o conceito de \textit{software} aberto. O autor descreve este conceito, como um \textit{software} o mais compatível possível com os hardwares com os quais se comunicaria e com os diferentes sistemas operacionais que podem executá-lo. Além disto, o \emph{software} deve ser modular, executado em módulos, de forma que o mal-funcionamento de uma de suas partes não impacte negativamente a execução das outras. Por fim, deve também ser escalável, e permitir a agregação de novos equipamentos e novas funcionalidades.

Frente à sua importância na área da automação, aspirantes a engenheiros devem se familiarizar com esta tecnologia. Atualmente existem ferramentas gratuitas focadas neste propósito, geralmente aplicadas no meio acadêmico, como descrita por \citeonline{Moraes2016}. Os chamados \emph{software} \emph{open source} são gratuitos, e seus códigos-fontes podem ser compartilhados, modificados e aplicados à vontade.

O programa ScadaBR, por exemplo, trata-se justamente de um sistema SCADA gratuito, construído em um servidor web \cite{silva2013}. O supervisório Elipse E3 também fornece uma versão de demonstração, com algumas limitações em relação à versão paga. Por fim, existem linguagens de programação \emph{open source}, como Python e C++, que possuem diversos materiais didáticos gratuitos disponíveis \emph{online}.

Tratando-se de plataformas pagas, exemplifica-se o \textit{software} de modelagens computacionais, utilizado na engenharia, chamado MATLab\textsuperscript{\tiny \textregistered}. Ele oferece um pacote voltado para estudantes que inclui o \emph{Simulink} para modelagem, simulação de processos, entre outras funcionalidades, e ferramentas para \emph{machine learning}. O valor deste conjunto em 03/12/20 era de USD 55,00 \cite{MathWorks}. No que concerne a supervisórios, o WinCC, fabricado pela Siemens é vendido por um de seus distribuidores, na versão Flexible por mais que R\$ 14.000,00 \cite{SHMR}.
	
Dados os valores de \textit{software} de engenharia, e a difusão de tecnologias \emph{open source}, existe uma necessidade e oportunidade de instituições de ensino aderirem à utilização de ferramentas gratuitas, desenvovidas pelos próprios estudantes. As vantagens não são somente financeiras, pois a construção destas ferramentas acrescenta profissionalmente a todos os envolvidos.

\section{Objetivos}

\subsection{Objetivos Gerais}

Desenvolver um sistema supervisório didático e gratuito capaz de se comunicar com dispositivos externos para construção de gráficos em sua interface e simulações em tempo real.

\subsection{Objetivos Específicos}

Com o foco no objetivo supracitado, foram levantados os seguintes objetivos específico direcionados desenvolvimento do supervisório didático:

\begin{itemize}
	\item Definir os requisitos de comunicação;
	\item Projetar e desenvolver a interface gráfica e os eventos que coordenam o funcionamento do sistema;
	\item Mapear e documentar o \textit{software} para facilitar futuras melhorias;
	\item Validar o funcionamento do supervisório com uma aplicação prática.
\end{itemize}

\section{Estrutura do Trabalho}

Este trabalho foi dividido em:

\begin{enumerate}
	\item No capítulo 1 o tema principal foi contextualizado de acordo com a tecnologia atual, o problema a ser solucionado foi apresentado e os objetivos foram esclarecidos.
	\item O capítulo 2 apresenta os conceitos básicos relativos aos assuntos que cercam o tema principal deste trabalho, contendo uma descrição curta dos sistemas SCADA existentes, das tecnologias que podem ser utilizadas para a criação de um supervisório e finalmente um resumo do método selecionado para a construção da ferramenta.
	\item O capítulo 3 trata da programação do supervisório em si, mostrando sua arquitetura e convenções empregadas com figuras e esquemas.
	\item O capítulo 4 toma o sistema pronto e o emprega em uma aplicação prática, com o intuito de validar seu funcionamento e exemplificar como o mesmo pode contribuir no ambiente acadêmico da universidade.
	\item No capítulo 5 este trabalho se encerra em forma de um texto conclusivo, onde, entre outros assuntos, são abordadas possíveis futuras melhorias ao sistema criado.
\end{enumerate}
\chapter{Fundamentação Teórica}

\section{Modelagem de Processos}

Segundo \citeonline{katsuhiro2010}, a dinâmica de muitos sistemas mecânicos, elétricos, térmicos, entre outros pode ser descrita em termos de equações diferenciais. Estas equações são obtidas pelas leis físicas que o regem, como a equação Bernoulli para dinâmca de fluidos. Para modelar um sistema por tais equações, levanta-se o nível de detalhes esperado, já que nem todas as variáveis de uma função tem impacto significativo na sua resposta. Um exemplo deste cenário seria a resistência do ar para corpos em queda livre, em alturas pequenas.

De acordo com \citeonline{Lathi2007}, um sistema, após modelado, pode traduzir-se em um sistema linear ou não linear. Um sistema é dito linear se o princípio da superposição se aplicar a ele. Este princípio afirma que a resposta produzida pela perturbação simultânea de mais de uma entrada é a soma cada resposta se calculada individualmente. Isto possibilita solucionar equações complexas a partir do cálculo de partes da mesma, simplificando a solução.

Segundo \citeonline{katsuhiro2010}, uma forma comum de representação de um sistema linear invariante no tempo é pelas chamadas funções de transferências. Sendo $y(t)$ a função que descreve uma das saídas de um processo, e $x(t)$ uma das entradas, a função de transferência $G(s)$ traduz a influência de x em y. Este formato sempre relaciona uma saída a uma entrada, e é sempre representado pela razão entre a transformada de Laplace da primeira sobre a segunda, ou:
\begin{equation}
\frac{\mathcal{L}(y)}{\mathcal{L}(x)} = \frac{y{(s)}}{x(s)}
\end{equation}

\subsection{Linearização de sistemas}

Enquanto que sistema lineares tem propriedades que facilitam sua modelagem, de acordo com \citeonline{corripio2006}, sistemas não lineares não possuem técnicas tão diretas para analisar suas dinâmicas. Assim, podem ser empregadas neles trasformações para sistemas lineares equivalentes.

Uma delas é a chamada linearização, que emprega a série de Taylor, truncanda no segundo termo, ou
\begin{equation}
f(x_1, x_2, .. , x_n) \simeq f({x_1}_0, {x_2}_0, ..., {x_3}_0) + \bigg( \sum_{i=1}^n \frac{df}{dxi}\left.\right|_{x_i = {x_i}_0} ({x_i} - {x_i}_0) \bigg)
\end{equation}
, sendo $x_i$ os parâmetros da função descritiva $f$, e $({x_i}_0)$ um ponto de operação. Obtém-se, assim, uma função linearizada em torno de um determinado ponto do processo . Quanto mais as variáveis do sistema linearizado se afastarem deste ponto de operação, maior será o erro deste sistema, em relação ao sistema gerador não linear.

\section{Controladores}

Segundo \citeonline{Lourenco2007}, não é possível determinar o tipo de controlador a se usar num determinada processo. Idealmente, o controlador mais simples, que satisfaça a "resposta desejada" deve ser ser escolhido. Porém, a escolha depende também das condições de operação do sistema e de performance, como o erro estacionário máximo, e o tempo de estabelecimento permitido.

%Mantendo-se neste raciocínio, serão comparados dois controladores distintos neste caso de teste. No primeiro caso, o sistema será linearizado em torno de um ponto de operação e o resultado será utilizado para sintonizar um controlador PID. Em seguida o controlador será acoplado ao sistema. No segundo caso, um controlador mais simples será empregado, o chamado LQR, sintonizado por uma equação matemática, descrita na sua respectiva seção neste documento.

\citeonline{corripio2006} descrevem o controle por realimentação como o emprego de um controlador que monitora uma variável (a variável controlada do processo), compara o valor lido com o valor desejado, o \emph{setpoint}, e computa o sinal de controle a ser enviado para o sistema, através de uma variável manipulada.

\subsection{PID}

Segundo \citeonline{katsuhiro2010}, a utilidade dos controladores do tipo PID (Proporcional, Integral e Derivativo) está na sua aplicabilidade geral à maioria dos sistemas. Por conta disto, é um controlador bem popular. Ele recebe como sinal o erro $e$ de um sistema, que é a diferença entra o valor da variável controlada e seu setpoint. A partir disto, calcula o sinal de controle como:
\begin{equation}
u(t) = K_p e(t) + K_i \int e(t) + K_d \frac{de(t)}{dt}
\label{eq_PID}
\end{equation}
, a partir de seus ganhos $K_p$, $K_i$ e $K_d$ (proporcional, integral, derivativo).

Ainda de acordo com \citeonline{katsuhiro2010}, a definição dos ganhos de um controlador é um processo chamado sintonia, e pode ocorrer por vários métodos. Para um PID, exemplificam-se Ziegler-Nichols, otimização e resposta em frequência. \citeonline{corripio2006} descrevem um outro método, denominado síntese: dada uma função de transferência conhecida $P(s)$ de primeira ordem para um determinado processo, a função de transferência entre o controlador e seu sinal de controle será, em malha fechada (Figura \ref{img_exemplo_processo}):
\begin{equation}
C(s) = \frac{u(s)}{e(s)} = \frac{1}{P(s)} \frac{1}{\tau_c s}
\end{equation}
, com $\tau_c$ sendo o único parâmetro de sintonização, e representando o tempo que o sistema controlado deve levar até atingir 62,3\% do seu estado estacionário. Assim, se 
\begin{equation}
P(s) = \frac{G}{\tau s + 1}
\end{equation}
, então
\begin{equation}
C(s) = K_p + K_i \frac{1}{s} = \frac{\tau}{G \tau_c} + \frac{1}{G \tau_c} \frac{1}{s}
\label{sintese_controlador}
\end{equation}
, sendo $\tau$ e $G$ o tempo de resposta e o ganho natural do sistema, respectivamente.

\begin{figure}[hbt]
	\centering
	\caption{Processo 1x1 com feedback e controlador}
	\includegraphics[width=0.8\textwidth]{exemplo_processo}
	\label{img_exemplo_processo}
\end{figure}

\subsection{LQR}

O Regulador Linear Quadrático, ou LQR, é um controlador de sintonia mais simples que o PID. Ele realiza um controle regulatório, ou seja, seu setpoint é sempre a origem (todos os estados e entradas em zero). Quanto à sua sintonia, segundo \citeonline{pythoncontrol}, o controlador recebe uma matriz $Q$ e uma matriz $R$, que atribuem pesos, respectivamente, aos estados e às entradas do sistema. Tomando um sistema genérico em espaço de estados:
\begin{equation}
\dot{x}(t) = A.x(t) + B.u(t)
\end{equation}
sendo $\dot{x}(t)$ as derivadas dos estados, $x(t)$ os estados, $A$ a matriz de estado, $u(t)$ as entradas e $B$ a matriz de entrada; o LQR aplicado será sintonizado de forma a minimizar a função quadrada de custo $J$ \ref{lqr_cost_func}:
\begin{equation}
J = \int_{0}^{\inf}(x'Qx + u'Ru)dt
\label{lqr_cost_func}
\end{equation} 

Segundo \cite{argentim2013}, a função de controle por realimentação $u$, para o caso do LQR, é representada pela equação \ref{lqr_generic_control_func}:
\begin{equation}
u = -Kx(t)
\label{lqr_generic_control_func}
\end{equation}
sendo $K$ a matriz de ganho do feedback dos estados.

Logo, em suma, a sintonia do LQR se dá ao encontrar a matriz $K$ que minimize a função de custo da equação \ref{lqr_cost_func}.

\section{SCADA}

De acordo com \citeonline{Martins2007}, os sistemas supervisórios podem ser considerados como o nível mais alto de IHM, pois mostram o que está acontecendo no processo e permitem ainda que se atue neste. A evolução dos equipamentos industriais, com a introdução crescente de sistemas de automação industrial, tornou complexa a tarefa de monitorar, controlar e gerenciar esses sistemas. 

Sistemas SCADAs são responsáveis por buscar informações de controladores e equipamentos diversos de automação, e manipular estas informações de acordo com o que foi programado. As aplicações mais simples se constituem na visualização dinâmica destes dados através de mostradores, cores, escalas, entre outros objeto gráficos em telas pré programadas. Uma estratégia, por exemplo, é a de literalmente desenhar todo um sistema produtivo através imagens já embutidas na biblioteca do software editor, e incluir nela todas as medições acessíveis, mapeando-as por onde se encontram no desenho. Outra seria a deagrupar as informações por temática, e mostrar diversas telas menos detalhadas, que alternem entre si de acordo com um temporizador interno. A Figura \ref{img_supervisorio_exemplo} ilustra um exemplo de tela para um sistema supervisório.

\begin{figure}[hbt]
	\centering
	\caption{Tela exemplo de um sistema supervisório}
	\includegraphics{Supervisorio_exemplo}
	Fonte: \href{https://www.agaads.com/service/scada-system/}{https://www.agaads.com/service/scada-system/}
	\label{img_supervisorio_exemplo}
\end{figure}

Por ser executado geralmente em um computador comum, a palavra chave de um sistema supervisório é flexibilidade. Um sistema SCADA deve ser capaz de se comunicar por diversos protocolos com diversos dispositivos, e adicionalmente disponibilizar os valores lidos para outros usuários, não somente os que têm acesso às telas. Isto se relaciona com o conceito já mencionado de software livre, descrito por \citeonline{junior2019}. Por possuir estas funcionalidades, software SCADAs ultrapassam o nível 2 na pirâmide de automação (Figura \ref{img_piramide_automacao}), chegando ao nível de supervisão da produção, pois agrupa as informações de diversos controladores e sensores em um só local, e gera suporte para ações de gerência.

\begin{figure}[hbt]
	\centering
	\caption{Pirâmide da automação}
	\includegraphics[width=0.8\textwidth]{piramide_automacao}
	Fonte: \href{https://www.logiquesistemas.com.br/blog/piramide-de-automacao-industrial/attachment/354/}{https://www.logiquesistemas.com.br/blog/piramide-de-automacao-industrial/attachment/354/}
	\label{img_piramide_automacao}
\end{figure}

Segundo \citeonline{Roggia2016}, dentre os principais benefícios do uso de sistemas de supervisão podem-se citar: informações instantâneas, redução no tempo de produção, redução no custo de produção, precisão das informações, detecção de falhas, aumento da qualidade e aumento da produtividade.

\section{Comunicação Serial}

Comunicação serial é um meio simples de dois equipamentos trocarem informação em formato de uma série de bits. De acordo com \citeonline{Bolton2015}, ela se dá através de um único cabo ou pino, transmitindo apenas um bit de cada vez. Apesar disto significar uma velocidade menor de envio de dados, a comunicação serial possui baixo custo, sendo popularmente utilizada na transmissão de dados por longas distâncias.Ainda existem outras tecnologias, como a chamada comunicação paralela, que utilizam mais canais de comunicação e podem portanto transmitir vários bits simultaneamente.

Segundo \citeonline{Mazidi2016}, existem dois tipos de comunicação serial: assíncrona ou síncrona. Como sugerido por seus nomes, a comunicação síncrona transmite blocos de tamanhos definidos, em momentos definidos, enquanto que o outro tipo transmite bytes de dados em qualquer momento. Para contornar a necessidade de escrever trechos de código que lidem como os dois casos, muitos fabricantes utilizam chips de circuitos integrados que manipulam o fluxo de dados, facilitando a escrita de scripts de comunicação.

Diversos equipamentos eletrônicos utilizam a comunicação serial, como mouses e teclados. O conhecido controlador Arduino UNO também permite a troca de bits por meio de suas portas seriais, de números 0 ou 1, ou uma mais comumente utilizada entrada USB (\cite{ArduinoSerial}). Ela também está presente em alguns protocolos de comunicação modernos, como Ethernet e Profibus.

Como ilustrado na figura \ref{img_serial_comm}, e fundamentado por \citeonline{Mazidi2016}, um modelo de transmissão para bytes (8 bits) de informação pela porta serial se constituiria em uma onda digital cujos formato se traduz em um bits 1 ou 0. O primeiro bit marca o início da transmissão, seguido por 8 bits relativos à informação enviada, um bit opcional de paridade (acrescentado por fins de validação de dados), e um último bit que encerra o bloco de informação. Protocolos de comunicação diferentes podem acrescentar ou remover características à sequência transmitida, seja para assegurar a integridade da informação, acrescentar outras informações na cadeia, ou para aumentar a velocidade de comunicação.

\begin{figure}[hbt]
	\centering
	\caption{Exemplo de transmissão serial de uma sequência de 3 bytes}
	\includegraphics{serial_comm} \\
	Fonte: \href{http://electrosofts.com/parallel/}{http://electrosofts.com/parallel/}
	\label{img_serial_comm}
\end{figure}

Em seu artigo, \citeonline{Denver1995} sugere o emprego de threads para contornar possíveis problemas na comunicação serial, como a espera para receber valores. Neste caso, threads impediriam que toda uma aplicação parasse até que a transmissão deste valor seja completada. 

\section{Qt em Python}

Segundo seu criador (informação verbal, \cite{Rossum2003}), Python foi criada no início da década de 90, com influências de outra linguagem na qual trabalhara, chamada ABC. O objetivo do projeto ABC era de criar uma linguagem que pudesse ser ensinada à usuários inteligentes de computadores, mas que não eram programadores nem desenvolvedores de softwares. Após ingressar em outro projeto, surgiu a necessidade de implementação de outra linguagem, onde Rossum teve a iniciativa de criar o Pyhton: uma  linguagem simples e escalável, que permitisse a contribuição de terceiros.

Se tratando de sua arquitetura, Python é uma linguagem de alto nível, orientada a objetos e com tipagem dinâmica e forte. As definições de escopo e blocos de código são representadas por indentações, o que torna o código mais organizado e visualmente aprazível, dispensando a utilização de chaves para delimitar escopo. Além disto, permite interoperabilidade com outras linguagens. Por exemplo, utilizando a ferramenta Cython é possível, a partir de um código Python, gerar um código equivalente em C. Existem funções, inclusive, que são desenvolvidas em C, a fim de agilizar o processamento de grandes bases de dados, mas implementadas em Python.

No âmbito acadêmico, Python apresenta boas vantagens. Não só é considerada simples fácil de aprender, como é gratuita e open source. Logo, seus usuários e clientes não têm custos com licenças e seus desenvolvedores podem usar livremente códigos publicados por terceiros, que geralmente se apresentam de fácil acesso na internet, e adaptá-los às suas necessidades. Em sua enquete, \citeonline{Overflow2019} avaliou Python como a 2ª “linguagem mais amada” pelo público, atrás de Rust.

Pela facilidade de compartilhamento e comunidade crescente de usuários, existem diversas bibliotecas úteis de Python que podem ser baixadas diretamente de um repositório online e facilmente instaladas. Como alguns exemplos, cita-se as libs SQLAlchemy \cite{sqlalchemy}, que permite criação e acesso a bancos de dados leves; NumPy, uma poderosa ferramenta para cálculos matriciais \cite{harris2020array}; e PyQt5, que possui objetos e métodos para criação de interfaces gráficas \cite{pyqtdoc}.

Segundo sua documentação \cite{pyqtdoc}, a biblioteca PyQt5 veio da biblioteca de C++ “Qt”, que implementa APIs para outras linguagens, permitindo-as implementar seus objetos em seus códigos. No seu site oficial, existe uma documentação extensa de todos os seus objetos em C++, e existem também muitos exemplos disponíveis online, tanto em sites oficiais do Qt como em fóruns de programadores.

Existem 3 módulos do PyQt utilizados para criação de GUIs locais:

\begin{itemize}
	\item QtWidgets: engloba os objetos gráficos principais, como botões, textos e layouts. O objeto genérico QWidget, herdado por diversos outros deste módulo, tem funções cruciais relativas ao posicionamento, geometria, visibilidade e estilo dos objetos gráficos.
	\item QtCore: este módulo lida com eventos dos objetos da GUI, como cliques de botões e edições de caixas de texto, e conecta estes eventos com funções definidas pelo programador. Ele também permite que ele configure a aplicação principal e coordena possíveis threads iniciadas por ela.
	\item QtGui: contém objetos que possibilitam a edição de cores e bordas dos Widgets e também lida com eventos relacionados à atualização da posição e estética destes objetos.
\end{itemize}

Para iniciar a GUI, deve ser criado um objeto \emph{QApplication}, que representa o núcleo da aplicação, juntamente com as janelas da mesma, que são objetos \emph{QMainWindow}. Estes aceitam um objeto genérico \emph{QWidget} como Widget central principal (através do método \emph{setCentralWidget()}), cujas características de tamanho e layouts internos definem o tamanho da janela. Inicia-se a aplicação pelo comando \emph{exec\_()}, e as janelas pelo comando \emph{show()}.

Numa interface criada com PyQt5, os objetos visíveis dispostos na tela são chamado de Widgets, e herdam da classe \emph{QWidget}. Por conta da arquitetura em objetos da biblioteca, o desenvolvimento de supervisório didático seguiu a filosofia de incluir um objeto dentro de outro. Assim, foi possível definir melhor o espaço e as condições de redimensionamento dos objetos quando a janela for esticada ou comprimida, pois cada layout redimensiona apenas os Widgets que contém, dentro do espaço disponível.

A Figura \ref{img_exemplo_qwidget} ilustra a organização padrão de uma aplicação em PyQt, listando também alguns objetos populares.

\begin{figure}[hbt]
\centering
\caption{Janela \emph{QMainWindow} com um \emph{QGroupBox} como Widget central, contendo vários outros Widgets em um layout \emph{QGridLayout} dentro de outro layout \emph{QHBoxLayout}}
\includegraphics{Exemplo_QWidget} \\
Fonte: \href{https://doc.qt.io/qt-5/qwidget.html}{https://doc.qt.io/qt-5/qwidget.html}
\label{img_exemplo_qwidget}
\end{figure}

\chapter{Desenvolvimento do Sistema Supervisório}

\section{Requisitos do Sistema}

Para a escolha da tecnologia empregada, foram levantados os seguintes requisitos do sistema:
\begin{itemize}
	\item Deve ser gratuito para o desenvolvimento, simples e de código aberto no intuito de permitir análise por partes de interessados e facilitar melhorias e expansões;
	\item Deve ser capaz de plotar gráficos em tempo real, advindos de no mínimo porta serial. Opcionalmente pode ser compatível com arquivos nos formatos .csv, .tsv, .xls, e .xlsx, ou permitir modelagem direta no software, por funções de transferência
	\item O software deve rodar no sistema operacional windows (mínimo Windows 7)
\end{itemize}

\section{Seleção das Tecnologias}

Pelo primeiro requisito, em relação à gratuidade da tecnologia, pensou-se primeiramente em utilizar plataformas abertas para sistemas supervisórios, como o ScadaBR ou a versão demo do Elipse E3. Já houveram trabalhos, inclusive, utilizando a primeira tecnologia (referenciados na bibliografia deste documento).

Como atestado em \cite{Moraes2016}, o ScadaBR é construído em um servidor web, utilizando geralmente o Apache TomCat (como servidor web). Assim, julgou-se necessário um certo período para acostumar-se com a sua programação, além de um conhecimento em redes para configuração da comunicação cliente-servidor.

Quanto à versão demo do Elipse E3, apesar do autor já possuir certo domínio da ferramenta, se trata de uma versão muito limitada. Segundo sua base de conhecimento, \cite{Elipse2019}, a versão demo permite somente até 20 tags de dados e a aplicação roda por um máximo de 2 horas, tendo que ser reiniciada manualmente após este período. Finalmente, não se sabe as implicações legislativas em utilizar o software para trabalhos acadêmicos, nem sua utilização contínua no âmbito da universidade, mesmo se tratando de uma versão de demonstração.

Por cumprir todos os requisitos, e por ser considerado um método inovador, foi decidido construir um sistema SCADA em Python. Desta maneira, o código-fonte da aplicação seria aberto, sua programação não exigiria grande esforço por parte do programador, e este trabalho contribuiria na difusão da implementação de softwares gratuitos e open source no meio acadêmico. Além disto, Python é uma linguagem contemporânea, tendendo a acompanhar o avanço tecnológico. Desta forma o sistema desenvolvido seria compatível com uma vasta gama de tecnologias atuais e futuras.

\subsection{Seleção das bibliotecas}

Como já mencionado, a linguagem Python possui inúmeras bibliotecas, para os mais variados fins. No decorrer da criação do sistema, foram utilizadas as seguintes bibliotecas no Quadro \ref{qdr_used_libs}.

\begin{quadro}
	\centering
	\begin{tabular}{|m{5em}|m{25em}|}
		\hline
		Biblioteca & Descrição \\
		\hline
		PyQt5 & Usada para a construção de GUIs, contém diversos objetos úteis como botões, caixas de texto e rótulos. Também trata do posicionamento e direção destes objeto nas janelas principal e periféricas \\
		\hline
		matplotlib & Contém ferramentas que permitem a plotagem e design de gráficos variados, inclusive com objetos backend que fazem uma ponte com GUIs construídas com PyQt5 \\
		\hline
		numpy & Biblioteca que lida com operações matriciais e cálculos avançados, com muitas funcionalidades similares ao MatLab. Também é capaz de gerar números aleatórios, que são úteis no teste do programa \\
		\hline
		pyserial & Permite a conexão com dispositivos externos pela porta serial e contém funções de escrita e leitura desta porta \\
		\hline
		python-control & Possibilita a criação de sistemas descritos em funções de transferência e espaço de estados, além de gerar a resposta simuladas para alguns formatos comuns de entradas, como degrau e impulso \\
		\hline
		pickle & Responsável pera serialização de objetos utilizados no código e armazenamento dos mesmos em um arquivo à parte. Lida também com a leitura e decodificação de objetos serializados \\
		\hline
	\end{tabular}
	\caption{Bibliotecas Python utilizadas no desenvolvimento do supervisório didático}
	\label{qdr_used_libs}
\end{quadro}

\section{Criando a interface gráfica}

A biblioteca PyQt funciona como uma linguagem orientada a objetos. Assim, esta arquitetura foi adotada no desenvolvimento da ferramenta. Foi utilizado uma folha de script que agrupasse a maior parte das classes implementadas, o form\_objects.py, e outro script principal, main.py que inicia a execução do programa. Um terceiro script, realtime\_objects.py, contempla objetos que rodam em tempo real e espera-se que futuros usuários também editem o código contido, por motivos explanados posteriormente neste documento.

No que concerne a interface gráfica do supervisório, imaginou-se um layout simplista. A aplicação conteria uma área para plotagem de gráficos, uma lista das séries de dados incluídas no programa pelos diversos métodos possíveis, e uma área para incluir séries novas. Os detalhes destes Widgets são listados abaixo, e sua disposição mostradas na Figura \ref{img_gui_macro}:

\begin{itemize}
	\item \emph{PlotManager}:representaria uma área de rolagem (QScrollArea) que conteria várias representações gráficas das séries de dados salvos na aplicação, com um pequeno preview destes dados, e botões para sua plotagem, edição e exclusão da série.
	\item \emph{MainPlotArea}: utiliza um objeto \emph{FigureCanvas}, da lib matplotlib, para plotagem detalhada das séries selecionadas na lista de séries. O eixo y seria adimensional, dependendo da variável medida e observada, e o eixo x representaria o tempo.
	\item \emph{DatasetConfig}: tem como Widget principal um seletor com abas (\emph{QTabWidget}), que permitiria a inclusão de séries de dados novas no programa, pelas fontes já mencionadas, sendo cada aba responsável pelas diferentes métodos (serial, arquivo, função de transferência, etc)
	\item \emph{SCADADialog}: se trata realmente de uma tela de supervisão, com uma área de plotagem atualizada em tempo real. Funciona somente quando os dados são importados por porta serial.
\end{itemize}

\begin{figure}[hbt]
	\centering
	\includegraphics[width=\textwidth]{GUI_macro}
	\caption{Supervisório didático e seus objetos principais}
	\label{img_gui_macro}
\end{figure}

Além dos Widgets supracitados, para simplificar o código e troná-lo mais prático, foi criado um objeto abstrato \emph{SeriesObject}. Ele armazena um agrupamento de série de dados que compartilham um mesmo eixo de tempo. Desta forma, foi mais fácil manipular as referências das séries pelo programa.

\section{\emph{Dataset Config}}

A principal função deste objeto é de interface da aplicação com outros dispositivos ou programas, com a finalidade de puxar séries de dados destas fontes. A segunda função é de formatar estas séries, atribuindo a elas um nome e um cabeçalho, além de definir que legenda aparece no gráfico principal quando a série for plotada.

Na fase de idealização do software, levantou-se possíveis meios para adicionar séries de dados novas no programa. Como se trata de um sistema supervisório, deve haver uma funcionalidade que permita o recebimento de dados externos, no mínimo por comunicação serial, bastante utilizada por controladores didáticos como Arduino e Raspberry Pi. Como adicionais,, seria útil também a importação de séries por arquivos \emph{.xls} ou \emph{.xlsx}. Por último, caso o usuário deseje utilizar funcionalidades não ofertadas pelo software didático, ele pode escrever seu próprio script Python e levar os resultados para o programa.

Na engenharia de Controle e Automação, é comum a prática de modelar um processa antes de observar seu comportamento empírico. Visto isso, o programa deveria oferecer uma maneira simples de simular processos diversos e plotá-los em contraste com o sistema real, proveniente do controlador. Logo, foi incluída biblioteca que simulasse as resposta de funções de transferência, por padrão a uma entrada degrau, e fornecidos os meios para que o usuário encadeie em série variadas funções, de parâmetros quaisquer, pelo objeto \emph{TransferFunctionConfig}.

Resumindo, existem quatro maneiras de importar ou gerar uma série de dados no software desenvolvido:

\begin{enumerate}
	\item \textbf{Por arquivo}, nos formatos \emph{csv} (Comma Separated Values), \emph{tsv} (Tab Separated Values), \emph{xls} (antigo arquivo Excel), \emph{xlsx} (arquivo Excel);
	\item \textbf{Por função de transferência}, informando o numerador e denominador de cada função, criadas em série e submetidas a uma entrada do tipo degrau, de valor definido pelo usuário;
	\item \textbf{Por comunicação serial}, configurando alguns parâmetros (porta, baud rate e tempo para timeout), separando cada valor enviado pelo dispositivo por uma tabulação e linhas por quebras de linha;
	\item \textbf{Por script Python}, escrevendo um código python funcional e retornando os valores das séries na ordem correta (lista dos valores das séries, série de valores de tempo, lista de nomes das séries, nesta ordem)
\end{enumerate}

Caso não haja problemas na importação e as condições de formatação para cada método de entrada forem satisfeitas, o programa abrirá uma caixa diálogo (\emph{QtWidgets.QDialog}) idêntica à da Figura \ref{img_edit_series_dialog} com uma preview dos dados e uma tabela com os valores numéricos. Através dele é possível editar cada valor separadamente, o nome da série e o cabeçalho.

\begin{figure}[hbt]
	\centering
	\includegraphics[width=\textwidth]{edit_series_dialog}
	\caption{\emph{ModelSeriesDialog}: Caixa diálogo para edição dos eixos e título das séries}
	\label{img_edit_series_dialog}
\end{figure}

A caixa diálogo contém um Widget bastante importante para a aplicação: o \emph{FigureCanvas}, proveniente de um módulo da biblioteca matplotlib, que funciona como um plugin para o PyQt. Apesar dele não ser um Widget do PyQt, ele contém, se não todos, uma boa parte de seus módulos e é tratado como tal para fins de posicionamento, geometria e inclusão nos layouts de tela. O \emph{FigureCanvas} faz uma interface com o objeto \emph{Figure}, responsável pela plotagem de gráficos de linhas, barras, etc, e permite que ele seja incluído numa GUI construída em PyQt.

O objeto \emph{Figure}, por sua vez, é bem similar ao objeto de mesmo nome no MATLab\textsuperscript{\tiny \textregistered}, aceitando métodos como \emph{plot()}, emph{add\_subplot()} e \emph{clear()}. Sua documentação completa, juntamente com a de outros objetos relevantes consta no site da biblioteca \href{https://matplotlib.org/}{matplotlib}. Quando embutido numa GUI por um \emph{FigureCanvas}, após a plotagem de gráficos e de formatações gerais, o segundo deve chamar o método \emph{draw()} para que seja atualizada a imagem.

%\textcolor{red}{Devo Focar no algoritmo de cada tipo de importação??}

\section{\emph{PlotManager}}

No lado direito da aplicação, se encontra uma lista das séries carregadas. Dado o espaço limitado, faz-se necessária uma área de rolagem. Assim, foi criado um objeto \emph{GraphicPlotList} que implementa esta área ao herdar de \emph{QtWidgets.QScrollArea}. Também existia originalmente um botão que criasse uma série de até 4 retas com inclinações aleatórias, para fins de teste.

Quando uma série nova é criada pelo \emph{DatasetConfig}, ela fica armazenada em um objeto \emph{SeriesObject}, que é representado graficamente na lista de séries \emph{GraphicPlotList} por um outro objeto \emph{GraphicPlotConfig} (Figura \ref{img_graphic_plot_config}).

\begin{figure}[hbt]
	\centering
	\includegraphics{graphic_plot_config}
	\caption{\emph{GraphicPlotConfig} não plotado em \emph{MainPlotArea}}
	\label{img_graphic_plot_config}
\end{figure}

A principal função deste objeto é identificar pelo nome e fonte a série que contém, e coordenar quando esta série deve ser plotada na área principal \emph{MainPlotArea}, além de permitir sua edição ou deleção. Contribuindo com a identificação dos dados, foi incluída uma visualização simples das séries, o que torna o visual estético. A cor de fundo do \emph{GraphicPlotConfig} alterna entre verde e vermelho, indicando se as séries contidas foram plotada ou não. Ao clicar no botão “Edit”, uma caixa diálogo idêntica à de incluir uma série nova aparece permitindo que o usuário edite seus valores, cabeçalho e nome.

Em PyQt, os objetos de uma GUI são “pintados” na tela por um objeto \emph{QtGui.QPainter}, de acordo com sua área, e paleta de cores. A primeira informação depende de alguns parâmetros, como o layout no qual ele está inserido ou sua política de tamanho (QSizePolicy). Isto ocorre no método \emph{paintEvent(event)}, chamado automaticamente quando o objeto é reposicionado ou quando sua aparência deve ser atualizada (cada caractere novo digitado numa caixa de texto, por exemplo).

Essa dinâmica torna necessário que o método seja sobrecarregado quando o programador deseje customizar a aparência de um botão ou o seu plano de fundo. A desvantagem é que, por sobrecarregar o método que pinta o objeto na interface, o objeto deve ser repintado manualmente. No caso de um botão, por exemplo, no mínimo o método \emph{drawRectangle()} e \emph{drawText()} deve ser invocado. Assim, quanto mais detalhado for um objeto, mais complicado se torna alterar suas cores e formatos internos. Existem maneiras que contornam a necessidade de criar um novo objeto e sobrescrever o método paintEvent, utilizando as chamadas stylesheets. Porém, para este trabalho, julgou-se mais simples a primeira opção, pelo fato dos objetos customizados possuírem um design pouco complexo.

O objeto \emph{GraphicPlotConfig}, por alternar suas cores de fundo, teve seu evento de "pintura" sobrecarregado. Como sua tela de fundo é representada apenas por um retângulo, sua implementação não foi muito dificultosa.

\section{\emph{MainPlotArea}}

Se tratando de análises de sistemas, a visualização dos dados é essencial. No canto inferior esquerdo da GUI, existe uma área de plotagem, implementada por um objeto \emph{FigureCanvas} (Figura \ref{img_main_plot_area}. Abaixo dele, uma faixa de ferramentas (\emph{NavigationToolbar2QT}) permite que o usuário edite algumas propriedades do gráfico plotado, aproxime ou distancie a imagem e, principalmente, salve a figura em formato de imagem. Na lista de séries à direita, ao clicar no botão “Plot”, a série associada serão desenhadas no Canvas, com legenda.

\begin{figure}[htb]
	\centering
	\includegraphics[width=0.8\textwidth]{main_plot_area}
	\caption{\emph{MainPlotArea}}
	\label{img_main_plot_area}
\end{figure}

\section{\emph{SCADADialog}}

Um sistema supervisório, como o apresentado neste documento, monitora dados de controladores geralmente industriais em tempo real. Por se tratar de um software didático, pensou-se em inclui no programa um suporte para futuras comunicações com um Arduíno ou qualquer outro microcontrolador ou até mesmo dispositivos que permitam comunicação serial. Assim, configurados os parâmetros de comunicação (\emph{baud rate}, nome da porta e tempo de timeout), o usuário pode plotar dados enviados por um controlador em tempo real, desde que os mesmos estejam no formato esperado: tabulações para separar diferentes séries de dados e quebras de linhas para finalizar um registro.

Ao importar dados por fonte serial no \emph{DatasetConfig}, sugere-se sempre testar a comunicação antes de clicar no botão “Puxar dados”. quando o mesmo é clicado, uma caixa diálogo \emph{DialogHeader} é aberta, para que o usuário edite os nomes das variáveis recebidas via serial. A primeira série é sempre o tempo, e o dispositivo conectado \textbf{deve} obedecer esta ordem. Ao clicar em OK nesta caixa, a comunicação com a porta serial será iniciada e uma janela de monitoramento surgirá, caso não tenha ocorrido nenhum erro de comunicação.

%Vale a pena colocar referencia nestes trechos
O código por trás da comunicação com periféricos implementa duas threads da biblioteca nativa emph{\_thread} do Python. Threads são trechos de código que rodam simultaneamente em um mesmo processo pai. Por conta disso, ao utilizar esta estrutura, alguns cuidados devem ser tomados pelo programador, no que se refere a acesso compartilhado à memória. Como não há controle de execução de cada thread separadamente, bugs podem ocorrer caso mais de um processo acesse e modifique o mesmo objeto ou variável simultaneamente.

Para contornar este problema, existem algumas estruturas que controlam o acesso de memória em programas que implementam threads, como monitores e semáforos. O último é, talvez, o mais trivial. Um semáforo possui dois estados: aberto ou fechado. Quando uma thread toma controle de um objeto no código, o semáforo é fechado, impedindo que outras threads façam o mesmo, até que o objeto seja liberado e o semáforo reaberto. Para Python, existem bibliotecas que implementam estruturas de controle, como a \emph{asyncio}, porém, por ser uma estrutura simples e não utilizada mais que algumas vezes no código-fonte deste trabalho, uma variável booleana bastou.

Após a conexão bem sucedida com o dispositivo, o programa escreve uma mensagem na porta serial (“go”, por padrão) e inicia suas duas threads, uma para ler dados da porta, e outra para atualizar o Canvas da caixa diálogo \emph{SCADADialog}. Esta separação se fez necessária pois o método do Canvas que o atualiza (emph{draw()}) é considerado custoso, e poderia travar a aplicação se fosse executada repetidas vezes. Outro motivo para implementação das threads é exemplificar a estudantes do código uma implementação simples de paralelismo, o que pode ser bastante util e que não é abordadas nos cursos tradicionais de Engenharia (excetuando computação).

Cada uma das threads tem um tempo de repetição, que define a periodicidade que elas executam suas instruções. O padrão definido foi de 0,1 segundos para ler dados da porta serial, e 1 segundo para atualizar o gráfico com os valores lidos. Quando a leitura da porta ocorre, os valores são armazenados em uma lista chamada emph{to\_be\_plotted}. Quando o gráfico é atualizado, os valores desta lista são movidos para outra, chamada emph{plotted}, que por sua vez é plotada, e o gráfico atualizado. A manipulação compartilhada da lista to\_be\_plotted justifica o emprego de um semáforo, pois uma thread escreve e outra lê.

Como já mencionado, o programa espera que os dados recebidos sejam espaçados por tabulações, separando diferentes variáveis, e quebras de linhas, separando diferentes leituras ao longo do tempo de execução. O primeiro valor lido sempre é considerado o tempo, e deve vir do dispositivo conectado, pois o mesmo é quem dita o andamento do processo controlado. Caso o número de variáveis numa mesma linha lida seja diferente do configurado no objeto DatasetConfig, o programa considerará que houve um erro e toda a linha de dados será ignorada.

A figura \ref{img_esquema_serial} esquematiza a execução do código de monitoramento serial:

\begin{figure}[hbt]
	\centering
	\includegraphics{esquema_serial}
	\caption{Esquema de leitura serial no supervisório didático}
	\label{img_esquema_serial}
\end{figure}

Para que o programa possa ser de fato implementado em aulas do curso, foi adicionada a possibilidade de envio de dados ao controlador a cada leitura da porta serial. Isto seria uma analogia a um controlador que atuasse num processo físico, medido por um dispositivo conectado. Devido à popularidade do microcontrolador Arduino, a implementação da rotina de controle foi projetada similarmente à sua programação. A rotina é feita em duas etapas, uma na função \emph{setup\_control}, chamada logo após uma conexão bem-sucedida, uma única vez, e outra na função \emph{loop\_control}, chamada logo após cada leitura da porta serial.

Numa aplicação de controle PID, por exemplo, a primeira função realizaria sua sintonia, enquanto que a segunda receberia como parâmetro a última linha lida da porta, calcularia a resposta do controlador PID, e a escreveria de volta. Obviamente, o dispositivo conectado deve ser programado para receber esta informação, o que requer certo conhecimento do usuário, tanto de Python como do dispositivo utilizado. Felizmente, códigos de exemplo se encontram disponíveis neste documento.

Durante o monitoramento e registro dos valores trazidos via serial, o gráfico irá atualizar numa janela de 20 segundos, contando do maior tempo registrado para trás. Isto se justifica no comportamento dinâmico da maioria dos sistemas, que atinge valores por vezes maiores que os estacionários. Esta medida impede que a escala do gráfico fique prejudicada, e variações pequenas relativas a um comportamento estacionário não sejam bem percebidas. O usuário pode, de todo modo, em tempo de execução clicar nos botões “Parar” e “Criar Série”, salvar as séries de dados lidas e plotá-las no objeto MainPlotArea, visualizando todo seu comportamento histórico.

\section{Salvamento Automático de Séries}

Durante a realização do caso de teste, descrito posteriormente neste documento, percebeu-se que no decorrer da análise de processos muitos ajustes são realizados, seja na função de controle da aplicação ou no controlador. Quando o script Pyhton é iniciado, uma cópia dele é criada e compilada, tornando impossível que alterações no script original em tempo de execução tenham influências no sistema. Por isso, o usuário tem que fechar o supervisório didático sempre que desejar alterar as funções \emph{setup\_control()} e \emph{loop\_control()}, o que resultaria na perda das séries já salvas pelo programa.

Para amenizar este problema, foi incluída uma funcionalidade de salvamento automático na aplicação. Uma das bibliotecas nativas do Python, o \emph{Pickle}, permite que um objeto ou variável do programa seja serializada em formato de arquivo, e salva em um diretório no computador. Isto ocorre através do comando \emph{pickle.dump()} Desta forma, o arquivo pode ser restaurado (\emph{pickle.load()}) e reincorporado ao código com os mesmos valores de atributos que possuía quando foi serializado.

Sempre que uma série é incluída, editada ou deletada, o arquivo "autosave.dat" é sobrescrito com a lista de séries da sessão (\emph{listSeries}). Em contrapartida, quando o objeto \emph{PlotManager} é inicializado, o arquivo "autosave.dat" é aberto e a lista de séries lá salva é restaurada.

\section{Diagrama de Relações entre Objetos}

A Figura \ref{img_diagrama_objetos} ilustra todos os objetos utilizados na construção do software, suas relações e principais métodos.

\begin{figure}[hbt]
	\centering
	\includegraphics[angle=90, width=\textwidth]{diagrama_objetos}
	\caption{Diagrama de relações entre os objetos empregados e funções principais}
	\label{img_diagrama_objetos}
\end{figure}
\chapter{Caso de Teste}

\section{Apresentação do Sistema}

Para ilustrar o funcionamento do sistema supervisório, foi utilizado um Arduino UNO que simulasse o funcionamento de um processo com dois tanques de área variável acoplados. A Figura abaixo o esquematiza:

\begin{figure}
	\centering
	\includegraphics{sistema_teste}
	\caption{Esquema de leitura serial no supervisório didático\\Fonte: Elaborado pelo Prof. Daniel Santana}
	\label{img_sistema_teste}
\end{figure}

Como percebido pelo esquema, os tanques têm o formato de tronco de cone e a variável controlada é a sua altura. O controle é feito por 2 bombas que aumentam ou diminuem a vazão de entrada de fluido em ambos os tanques. A medição é realizada por dois sensores de nível, sujeito a ruídos brancos.

As equações deste processo são descritas abaixo:
\\\\
$
\frac{dh_1}{dt} = \frac{1}{\beta(h_1)}(u_1 - k\sqrt{\rho g h_1} + k\sqrt{\rho g h_2})\\
\frac{dh_2}{dt} = \frac{1}{\beta(h_2)}(u_2 - k\sqrt{\rho g h_2})\\
\beta(h_i) = frac{dV}{dh_i}\\
V(h_i) = \frac{\pi\gamma^2}{3}(h_i + \frac{B}{2\gamma})^3 - \frac{\pi}{3\gamma}(\frac{B}{2})^3\\
\gamma = \frac{A-B}{2h_M}\\
$

Seus parâmetros e variáveis são descritos e dimensionados na tabela abaixo:

\begin{center}
	\begin{tabular} {|m{5em}|m{15em}|m{8em}|}
		\hline
		Símbolo & Descrição & Valor (u.m.) \\
		\hline
		A & diâmetro superior & 4 ($m$) \\
		B & diâmetro inferior & 1 ($m$) \\
		hm & altura máxima & 4 ($m$) \\
		V & volume & - ($m^3$) \\
		h & altura & - (m) \\
		$\rho$ & densidade do fluido & 1000 ($kg/m^3$) \\
		g & aceleração da gravidade & 9,8 ($m/s^2$) \\
		k & constante de descarga no tanque & 0,001 (-)\\
		u & vazão da bomba & - ($m^3/s$)\\
		\hline
	\end{tabular}
\end{center}

\section{Sintonia do Controlador}

Por tratar-se de um sistema não linear, serão comparados dois controladores sintonizados por métodos distintos. No primeiro caso, o sistema será linearizado em torno de um ponto de operação e o resultado será utilizado para sintonizar um controlador PID. Em seguida o controlador será acoplado ao sistema. No segundo caso, será utilizada uma lei de controle que anula a não-linearidade do sistema, transformando-o em uma sistema linear. Após este passo, também será utilizado um controlador PID para controlá-lo.

\subsection{Linearização do Sistema}

O processo de linearização de um sistema consiste na aplicação da série de Taylor em suas equações descritivas num determinado ponto de operação.

\begin{figure}
	\centering
	$
	f(x1, x2, .. , xn) \simeq f(x1_0, x2_0, ..., x3_0) + \bigg( \sum_{i=1}^n \frac{df}{dxi}\big|_{xi=xi_0} (xi - xi_0) \bigg)
	$
	\caption{Série de Taylo truncada na primeira ordem}
\end{figure}

sendo a expressão $(xi - xi_0)$ chamada de desvio e representada por $\overline{xi}$.

Aplicando a linearização no sistema de estudo, e tomando como ponto de operação o sistema em um estado estacionário $h_{1_0}=h_{1_{ss}}=1, h_{2_0}=h_{2_{ss}}=1, u_{1_0}=u_{1_{ss}}=0, u_{2_0}=u_{2_{ss}}=k\sqrt{\rho g}$. As novas equações do sistema, em desvio, serão então:

$
\frac{dh_1}{dt} = -\big(\frac{d\beta^{-1}(h)}{dh}\bigg|_{{h_1}_{ss}} k\sqrt{\rho g {h_1}_{ss}} + \frac{\beta^-1({h_1}_{ss}) k\sqrt{\rho g}}{2\sqrt{{h_1}_{ss}}}\big)\overline{h_1} + \big(\frac{d\beta^{-1}(h)}{dh}\bigg|_{{h_2}_{ss}} k\sqrt{\rho g {h_2}_{ss}} + \frac{\beta^-1({h_2}_{ss}) k\sqrt{\rho g}}{2\sqrt{{h_2}_{ss}}}\big)\overline{h_2} +  \beta^{-1}({h_1}_{ss})\overline{u_1}
$

$
\frac{dh_2}{dt} = -\big(\frac{d\beta^{-1}(h)}{dh}\bigg|_{{h_2}_{ss}} k\sqrt{\rho g {h_2}_{ss}} + \frac{\beta^-1({h_2}_{ss}) k\sqrt{\rho g}}{2\sqrt{{h_2}_{ss}}}\big)\overline{h_2} + \beta^{-1}({h_1}_{ss})\overline{u_2}
$

$
\beta^{-1}(h) = \frac{4}{\pi}\frac{1}{(\gamma h)^2 + B\gamma h + B^2} 
$


\backmatter

% \bibliographystyle{abnt-alf}
% INCLUA AQUI O SEU ARQUIVO DE REFERÊNCIA
%\bibliography{References_rev02.bib}

\appendix
\addcontentsline{toc}{chapter}{Anexos e Apêndices}

\include{ApendiceXX}

%\includepdf{CAPA_DISSERTACAO_PEI_DANIEL_DINIZ_SANTANA_FUNDO.pdf}

\end{document}
