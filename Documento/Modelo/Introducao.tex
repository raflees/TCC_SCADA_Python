\chapter{Introdução} \label{Chap:Introdução}

\section{Contextualização}

Incluir

\section{Problema e Justificativa}

Frente à importância de sistemas supervisórios no âmbito da Automação e Controle de processos, e dos altos custos de licenças de alguns softwares como MatLab e SIMATIC WinCC, surgiu a ideia da construção de uma plataforma que utilize uma linguagem gratuita e open-source e sirva de alternativa a aplicações deste ínterim. A mesma seria utilizada para comunicar-se com qualquer controlador conectado por porta serial ao computador que a executasse, registrando em forma de gráficos e valores estatísticos variáveis inerentes a sistemas mecânicos, elétricos ou quaisquer outros monitorados por sensores conectados a tal controlador, ou modelados por funções de transferências. Além disto, teria código livre e aberto, tanto para o estudo e aprendizado dos estudantes da universidade, como para futuras melhorias e inclusão de novas funcionalidades.

\section{Objetivos}

\subsection{Objetivos Gerais}

Desenvolver em linguagem Python uma GUI (Graphical User Interface) capaz de se comunicar com dispositivos externos que lhe fornecessem dados numéricos para plotagem em sua interface, de forma que seja possível a análise de respostas de sistemas físicos diversos e comparação com modelagens criadas pelos usuários deste sistema.

\subsection{Objetivos Específicos}

Com o foco nos objetivos supracitados, foram levantados os seguintes pequenos milestones a serem alcançados no decorrer do desenvolvimento do sistema:

\begin{itemize}
	\item Definir os requisitos de comunicação
	\item Projetar a(s) tela(s) do sistema
	\item Programar os eventos que coordenam o funcionamento do sistema
	\item Configurar a comunicação com dispositivos externos (preferencialmente tomando o controlador Arduino como base)
\end{itemize}

\section{Estrutura do Trabalho}

Este trabalho foi divididos em:

\begin{enumerate}
	\item No capítulo 1 o tema principal foi contextualizado de acordo com a tecnologia atual, o problema a ser solucionado foi apresentado, e os objetivos foram esclarecidos e segmentados em um passo-a-passo.
	\item No capítulo 2 consta uma fundamentação teórica relativa aos assuntos que cercam o tema principal deste trabalho, a iniciar por uma descrição curta dos softwares SCADA existentes, das tecnologias que podem ser utilizadas para a criação de uma GUI e finalmente um resumo da linguagem selecionada para a construção da interface
	\item O capítulo 3 trata da programação da plataforma em si, em linguagem Python, mostrando sua arquitetura e convenções empregadas com figuras e esquemas.
	\item O capítulo 4 apresenta o sistema já pronto e aborda as lições aprendidas no decorrer de seu desenvolvimento, além de outras discussões relativas à sua programação.
	\item No capítulo 5 este trabalho se encerra em forma de um texto conclusivo, onde, entre outros assuntos, são abordadas possíveis futuras melhorias ao sistema criado.
\end{enumerate}